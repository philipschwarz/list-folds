\begin{frame}
\frametitle{Summary}
\begin{block}{the key intuition}
\begin{itemize}
\item left fold performs a \emph{loop}, just like we are familiar with
\item right fold performs \emph{constructor replacement}
\end{itemize}
\end{block}
\end{frame}

\begin{frame}
\frametitle{Summary}
\begin{block}{from this we derive some observations}
\begin{itemize}
\item left fold will \emph{never} work on an infinite list
\item right fold \emph{may} work on an infinite list
\item These observations are independent of specific programming languages
\end{itemize}
\end{block}
\end{frame}

\begin{frame}[fragile]
\frametitle{Summary}
\begin{block}{from this we also solve problems}
\begin{itemize}
\item \lstinline[mathescape]{product = $\ldots$}
\item \lstinline[mathescape]{append = $\ldots$}
\item \lstinline[mathescape]{map = $\ldots$}
\item \lstinline[mathescape]{length = $\ldots$}
\item \lstinline[mathescape]{$\ldots$}
\end{itemize}
\end{block}
\end{frame}

\begin{frame}
\frametitle{Summary}
\begin{block}{}
\begin{itemize}
\item intuitively, this is precisely what list folds do
\item this intuition is \textbf{precise} and requires no footnotes
\end{itemize}
\end{block}
\end{frame}

\begin{frame}
\frametitle{Summary}
\begin{center}
THE END
\end{center}
\end{frame}
